% Chapter 1

\chapter{Motivation} % Write in your own chapter title
%\label{fig:INTRODUCTION}
%\lhead{CHAPTER 1. \emph{INTRODUCTION}} % Write in your own chapter title to set the page header
	To bug is human, to debug is divine. Software evolution has high associated costs and effort. A survey by the National Institute of Standards and Technology estimated that the annual cost of software bugs is about \$59.5 billion. Some software maintenance studies indicate that maintenance costs are atleast 50\% and sometimes more than 90\%, of the total costs associated with a software product. These surveys suggest that making the bug fixing process more efficient would reduce evolution effort and lower software production costs.
	
	Most software projects use bug trackers to organize the bug fixing process and facilitate application maintenance. For instance, Bugzilla is a popular bug tracker used by many large projects, such as Mozilla, Eclipse, KDE, and Gnome. These applications receive hundreds of bug reports a day; ideally, each bug gets assigned to a developer who can fix it in the least amount of time. This process of assigning bugs, known as bug assignment, is complicated by several factors: if done manually, assignment is labour-intensive, time-consuming and fault-prone; moreover, for open source projects, it is difficult to keep track of active developers and their expertise. Identifying the right developer for fixing a new bug is further aggravated by growth, e.g., as projects add more components, modules, developers and testers, the number of bug reports submitted daily increases, and manually recommending developers based on their expertise becomes difficult.
	
	Reports indicate that, on average, the Eclipse project takes about 40 days to assign a bug to the first developer, and then it takes an additional 100 days or more to reassign the bug to the second developer. Similarly, in the Mozilla project, on average, it takes 180 days for the first assignment and then an additional 250 days if the first assigned developer is unable to fix it. These numbers indicate that the lack of effective, automatic assignment and toss reduction techniques results in considerably high effort associated with bug resolution. Reassigning a bug to another developer if the previous assignee is unable to resolve it is known as Bug Tossing. It can be inferred from the dataset of Eclipse that almost 90\% of all "Fixed" bugs have been tossed at least once.

